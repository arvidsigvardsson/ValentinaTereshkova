% Created 2017-05-10 ons 11:35
% Intended LaTeX compiler: pdflatex
\documentclass[11pt]{article}
\usepackage[utf8]{inputenc}
\usepackage[T1]{fontenc}
\usepackage{graphicx}
\usepackage{grffile}
\usepackage{longtable}
\usepackage{wrapfig}
\usepackage{rotating}
\usepackage[normalem]{ulem}
\usepackage{amsmath}
\usepackage{textcomp}
\usepackage{amssymb}
\usepackage{capt-of}
\usepackage{hyperref}
\author{arvid}
\date{\today}
\title{}
\hypersetup{
 pdfauthor={arvid},
 pdftitle={},
 pdfkeywords={},
 pdfsubject={},
 pdfcreator={Emacs 24.5.1 (Org mode 9.0.6)}, 
 pdflang={English}}
\begin{document}

\tableofcontents

\section{Koordinater Fälttest Niagara 2017-05-10}
\label{sec:org589221e}
\begin{itemize}
\item Punkt 1 har koordinaterna (34.892477919403944, 50.03513750000002)
\item Punkt 2 har koordinaterna (136.87024924572668, 200.42411249999998)
\item Punkt 3 har koordinaterna (56.452925984231165, 362.6321375)
\item Punkt 4 har koordinaterna (300.6973100335595, 196.11763749999997)
\item Punkt 5 har koordinaterna (451.77866876725153, 350.3383000000001)
\item Punkt 6 har koordinaterna (434.0392809681631, 35.63919999999991)
\item Kamerans position: (278.9155267924208, 134.61136249999998)
\end{itemize}
\end{document}
